Insgesamt ist die Softwarelösung in die drei Module Benutzeroberfläche, Kommunikation und Datei-Ein-/Ausgabe aufgeteilt. Grund dafür ist, dass insbesondere die Module Kommunikation und Datei-Ein-/Ausgabe auch unabhängig voneinander arbeiten können.

So wird die Kohäsion der einzelnen Teile erhöht und die Kopplung verringert. Im foldenden werden die einzelnen Teile genauer erklärt.

\subsection{Benutzeroberfläche}
architektur und so
% TODO @Lars

\subsection{Kommunikation}
Das Modul Kommunikation ist für den Verbindungsaufbau und den Nachrichtenaustausch zwischen den Parteien zuständig. Nach außen soll es eine stark abstrahierte Schnittstelle bieten, die Fehlbenutzung vermeidet und die Verwendung maximal vereinfacht.

Das Modul soll in die Untermodule \textit{Client} und \textit{Protokoll} aufgeteilt werden. Das Modul \textit{Client} bietet dabei die Schnittstellen für das Senden und Empfangen von Nachrichten. Mithilfe des \textit{Protokoll} Moduls wird die Möglichkeit geboten die Verbindung zweier \textit{Client} Instanzen aufzubauen und zu verschlüsseln.

Insgesamt sollen drei Umsetzungen eines Clients implementiert werden. Darunter fallen zwei Basisimplementierungen, die die Kommunikation über TCP und UDP ermöglichen. Die dritte Implementation baut auf einer der Basisimplementierungen auf und verschlüsselt die entsprechende Kommunikation.

Für jeden Client besteht jeweils die Möglichkeit den Sender- und Empfängerteil in zwei seperate Objekte aufzuteilen. Somit wird sichergestellt, dass Anwender die Kommunikation auf mehrere Threads aufteilen können.

Das Protokoll verwendet die implementierten Clients und sorgt für eine reibungslosen Übergangt zwischen den verschiedenen Implementationen. Dies ist notwendig, da zunächst nur eine UDP-Kommunikation aufgebaut werden kann, da das TCP-Protokoll eine zeitliche Synchronisation benötigt. Diese Synchronisation kann nur über eine bereits existierende Verbindung erfolgen.

Um die Verbindung während des Aufbaus der TCP-Verbindung abzusichern ist der erste Schritt nachdem UDP-Verbindungsaufbau die Verschlüsselung der Verbindung. Anschließend erfolgt ein Versuch die Zeiten zu synchronisieren und eine TCP-Verbindung aufzubauen. Bei Erfolg wird die Kommunikation in eine verschlüsselte TCP-Verbindung überführt, andernfalls wird die bereits bestehende UDP-Verbindung weiter verwendet.

Benutzern der Schnittstelle ist offengestellt welche Art der Kommunikation sie verwenden möchten. Das Protokoll kann zu jedem Zeitpunkt in einen aktiven Client überführt werden, über den dann entsprechnder Datenaustausch möglich ist.

\subsection{Datei-Ein-/Ausgabe}
architektur und so
% TODO @Simon