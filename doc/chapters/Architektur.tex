\label{sc:architektur}

Insgesamt ist die Softwarelösung in die drei Module Benutzeroberfläche, Kommunikation und Datei-Ein-/Ausgabe aufgeteilt. Der Grund dafür ist, dass insbesondere die Module Kommunikation und Datei-Ein-/Ausgabe auch unabhängig voneinander funktionieren können.

Dadurch wird die Kohäsion der einzelnen Module erhöht und die Kopplung verringert. Im Folgenden werden die einzelnen Module genauer erklärt.

\subsection{Benutzeroberfläche}
Die Benutzeroberfläche soll einfach und modern gestaltet sein. Sie soll hauptsächlich auf eine Farbpalette aus Schwarz, Weiß und Grau zurückgreifen. Darüber hinaus soll die Benutzeroberfläche auf allen Plattformen einheitlich und konsistent aussehen.
Dies gewährleistet eine intuitive Navigation und problemlose Ausführung von Aktionen für die Benutzer. Die Farbgestaltung in Schwarz, Weiß und Grau soll eine zeitlose Ästhetik bieten und die Konzentration auf den Inhalt fördern. Die Einheitlichkeit der Benutzeroberfläche über verschiedene Plattformen hinweg soll eine nahtlose und vertraute Nutzungserfahrung bieten, unabhängig davon, ob die Anwendung auf Mobilgeräten, Tablets oder Desktop-Computern verwendet wird.
Durch die Umsetzung dieser Anforderungen wird die Benutzerfreundlichkeit optimiert und eine effiziente Nutzung der Anwendung ermöglicht.


\subsection{Kommunikation}
Das Modul “Kommunikation“ ist für den Verbindungsaufbau und den Nachrichtenaustausch zwischen den Parteien zuständig. Nach außen hin soll es eine stark abstrahierte Schnittstelle bieten, die Fehlbenutzung vermeidet und die Verwendung maximal vereinfacht.

Das Modul soll in die Untermodule “Client“ und “Protokoll“ aufgeteilt werden. Das Modul “Client“ bietet dabei die Schnittstellen für das Senden und Empfangen von Nachrichten. Mithilfe des “Protokoll“-Moduls wird die Möglichkeit geboten, die Verbindung zweier “Client“-Instanzen aufzubauen und zu verschlüsseln.

Insgesamt sollen drei Implementierungen eines Clients entwickelt werden. Darunter fallen zwei Basisimplementierungen, die die Kommunikation über TCP und UDP ermöglichen. Die dritte Implementierung baut auf einer der Basisimplementierungen auf und verschlüsselt die entsprechende Kommunikation.

Für jeden Client besteht jeweils die Möglichkeit, den Sender- und Empfängerteil in zwei separate Objekte aufzuteilen. Dadurch wird sichergestellt, dass Anwender die Kommunikation auf mehrere Threads aufteilen können.

Das Protokoll verwendet die implementierten Clients und sorgt für einen reibungslosen Übergang zwischen den verschiedenen Implementierungen. Dies ist notwendig, da zunächst nur eine UDP-Kommunikation aufgebaut werden kann, da das TCP-Protokoll eine zeitliche Synchronisation benötigt. Diese Synchronisation kann nur über eine bereits bestehende Verbindung erfolgen.

Um die Verbindung während des Aufbaus der TCP-Verbindung abzusichern, ist der erste Schritt nach dem UDP-Verbindungsaufbau die Verschlüsselung der Verbindung. Anschließend wird versucht, die Zeiten zu synchronisieren und eine TCP-Verbindung aufzubauen. Bei Erfolg wird die Kommunikation in eine verschlüsselte TCP-Verbindung überführt, andernfalls wird die bereits bestehende UDP-Verbindung weiterverwendet.

Benutzern der Schnittstelle steht es offen, welche Art der Kommunikation sie verwenden möchten. Das Protokoll kann zu jedem Zeitpunkt in einen aktiven Client überführt werden, über den dann entsprechender Datenaustausch möglich ist.

\subsection{Datei-Ein-/Ausgabe}
Das I/O-Modul ist dafür zuständig, Dateien aufzuteilen und zusammenzuführen sowie Nachrichten zu kodieren, die zwischen den Clients ausgetauscht werden. Um eine klare Struktur zu gewährleisten, wird es in mehrere Untermodule aufgeteilt.

Das “Error“-Modul hilft dabei, verschiedene Fehlerarten abzufangen und zu behandeln.

Das “Hash“-Modul sorgt dafür, dass die Integrität der übertragenen Daten gewahrt bleibt.

Das “Order“-Modul erleichtert die Verwaltung von Bestellungen und den Informationsaustausch zwischen den Clients.

Das “Offer“-Modul hat ähnliche Funktionen wie das “Order“-Modul, jedoch für Angebote.

Das “File“-Modul greift auf das Dateisystem zu, liest die Daten ein, teilt sie auf und fügt sie wieder zusammen. Durch die Aufteilung der Dateien in mehrere Stücke soll der Übertragung großer Datenmengen erleichtert werden.

Das “General“-Modul enthält alle weiteren Funktionen, die im System benötigt werden. Es dient als eine Art “Catch-all“-Modul für alle Funktionen und Prozesse, die nicht in die anderen spezifischeren Module passen.
