Im folgenden sind die Anforderungen an die Software definiert. Die Anforderungen werden dabei in funktionale und nicht funktionale Anforderungen unterteilt.

\subsection*{Funktionale Anforderungen}
\begin{description}
\item[Anforderung 1: Lauffähigkeit] Die Software soll auf allen gängigen Betriebssystemen lauffähig sein.
\item[Anforderung 2: Übertragung] Die Software soll es ermöglichen, ohne die Verwendung eines Servers, Daten direkt zwischen zwei Parteien zu übertragen.
\item[Anforderung 3: Sicherheit] Die gesamte Kommunikation zwischen den Parteien soll verschlüsselt erfolgen.
\item[Anforderung 4: Dateien] Die Software soll die Möglichkeit bieten, Dateien zwischen den Parteien auszutauschen. Diese sollen automatisch aus dem eingelesen und abgespeichert werden.
\item[Anforderung 5: Programmiersprache] Die Software muss mit der Programmiersprache Rust umgesetzt werden.
\item[Anforderung 6: Zuverlässigkeit] Die Kommunikation zwischen den Parteien muss auf eine zuverlässige Art und Weise erfolgen. Die Pakete müssen reihenfolgengetreu und verlustfrei empfangen werden können.  
\end{description}

\subsection*{Nicht-Funktionale Anforderungen}
\begin{description}
\item[Anforderung 1: Benutzerfreundlichkeit] Die Software soll eine leicht verständliche Oberfläche bieten. Zu jedem Zeitpunkt muss klar seine welche Aktionen möglich sind. Es werden keine überflüssigen Informationen angezeigt.
\end{description}