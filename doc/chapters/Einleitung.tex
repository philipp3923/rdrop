In der modernen Welt ist der Austausch von Daten ein immer wichtiger werdender Bestandteil des alltäglichen Lebens. Dennoch bieten die meisten Anbieter für das Übertragen von Daten keine Möglichkeit, diese auf direktem Wege zwischen den Nutzern auszutauschen. In den meisten Fällen müssen die Daten zunächst auf einem Server zwischengespeichert werden, damit die empfangende Partei diese Daten herunterladen kann.

Dieser Umweg ist nicht nur zeitaufwändig, sondern auch ein Sicherheitsrisiko, da die Anbieter, die die Daten zwischenspeichern, Zugriff auf die Daten erhalten \cite{bsi-cloud}. Eine Alternative dazu bietet AirDrop. AirDrop ist ein Dienst der Firma Apple, der es ermöglicht, Daten direkt zwischen zwei Geräten zu übertragen. Jedoch besitzt AirDrop den entscheidenden Nachteil, dass sich die Geräte in unmittelbarer Nähe zueinander befinden müssen \cite{apple-airdrop}.

Alle bisherigen Lösungen benötigen mindestens einen Server, um die Kommunikation zwischen beiden Parteien aufzubauen. Eine wirklich einfache Lösung für die gesicherte direkte Übertragung von Daten zwischen zwei Parteien existiert bisher nicht.

Ziel dieses Projekts ist es, eben dieses Problem zu lösen. Dafür wird eine Anwendung entwickelt, die es ermöglicht, ohne die Verwendung eines Servers Daten auf direktem Wege zwischen zwei Parteien auszutauschen. Die Anwendung soll dabei auf allen gängigen Betriebssystemen lauffähig sein. Insbesondere wird ein Fokus auf die unkomplizierte Bedienung der Anwendung gelegt, um auch unerfahrenen Nutzern die Möglichkeit zu bieten, Daten sicher und schnell auszutauschen.

Das folgende Dokument beinhaltet die Dokumentation des Projekts. Dabei werden zunächst die Anforderungen an die Anwendung definiert. Anschließend wird die Architektur der Anwendung erklärt und wie diese umgesetzt wurde. Schlussendlich wird die Umsetzung und Lösung des Projekts kritisch reflektiert und ein Ausblick auf mögliche Erweiterungen gegeben.