Dieses Kapitel beschreibt die Umsetzung der in Kapitel \ref{sc:architektur} beschriebenen Architektur. Dabei wird auf die einzelnen Module eingegangen und die implementierten Funktionen erläutert. 

Die Module Kommunikation und Datei-Ein-/Ausgabe werden dabei als Bibliotheken implementiert, die von der Benutzeroberfläche verwendet werden. Die Benutzeroberfläche wird als eigenständiges Programm implementiert.

\subsection{Benutzeroberfläche}
Bei der Umsetzung der Benutzeroberfläche für rdrop stand eine einfache, intuitive und benutzerfreundliche Erfahrung im Vordergrund. Um dies zu erreichen, kamen verschiedene Technologien zum Einsatz, darunter Tauri, ein Framework für plattformübergreifende Anwendungen, NextJS, ein Framework für servergerenderte React-Anwendungen, und Figma, ein kollaboratives Interface-Design-Tool.

Tauri ist ein GUI Framework, das es ermöglicht, native Anwendungen aus Web-Technologien wie HTML, CSS und JavaScript zu erstellen. Tauri bietet eine einfache und intuitive API, die es ermöglicht, native Funktionen wie Dateisystemzugriff, Benachrichtigungen und Systemdialoge zu verwenden. Darüber hinaus bietet Tauri eine einfache Möglichkeit, die Anwendung für verschiedene Plattformen zu erstellen, einschließlich Windows, Linux und MacOS.

Für die Website die in der Applikation gerendert wird wurde React gewählt. React bietet durch seine Komponentenstruktur eine einfache und intuitive Entwicklung. Beispielsweise können einzelne UI-Komponenten wie bespielswiese ein Textfeld in einer eigenen Datei erstellt werden. Diese Komponente kann dann in anderen Komponenten verwendet werden. Dies ermöglicht eine einfache Wiederverwendung von Komponenten und eine einfache Strukturierung der Anwendung. NextJS ist eine Erweiterung von React, welches eigentlich im verwendet wird im Server-Side-Rendering zu nutzen. Wir haben NextJS trotzdem gewählt, da dieses über viele weitere Funktionen verfügt, wie beispielsweise Routing. Mittels Static Site Generation konnte die Website dann in statisches HTML umgewandelt werden, welches von Tauri verwendet werden kann. Server-Side-Rendering wurde nicht verwendet, da dies mit Tauri nicht möglich ist und keine Vorteile bietet. Dieses würde nur die Ladezeit der Website erhöhen, welches in einem lokalen Kontext nicht notwendig ist, da die Website nicht über das Internet geladen werden muss.

Figma wurde verwendet, um das Design der Benutzeroberfläche zu erstellen. In Figma wurden einige Mockups erstellt um das grobe Layout der Anwendung zu planen. Zudem wurden auch die Farben und Schriftarten der Andwendung bestimmt. Änderungen konnten somit leichter vorgenommen werden, da diese nicht im Code vorgenommen werden mussten.

Um das Design der Anwendung zu vereinheitlichen wurde Material Design 3 als Referenz verwendet. Material Design ist eine Spezifikation von Google um Applikationen in einem Einheitlichen Design zu erstellen. Für die wichtigsten Elemente einer Andwendung wie Buttons, Textfelder und Schriftarten werden Größe, Abstand und andere Guidelines vorgegeben. Auch die in der Anwendung verwendeten Icons sind die Material Design Icons. Diese wurden als Schriftart in der Website eingebunden.

\subsection{Kommunikation}
Für die Umsetzung der Clients werden die notwendigen Funktionalitäten, die an einen Client gestellt werden, in Traits festgelegt. Jeder der Clients implementiert diese. Somit können alle der Clients im generischen Kontext austauschbar verwendet werden.

Die Implementierung des TCP-Clients stellt dabei keine besondere herausforderung dar. In Addition zu dem standardmäßigen Protokoll muss lediglich die Länge eines Pakets hinzugefügt werden. Dies ist notwendig, da in einem TCP-Stream keine Paketgrenzen vorhanden sind.

Nach der 6. Funktionalen Anforderung muss die Kommunikation zuverlässig erfolgen. Aus diesem Grund muss für den UDP-Client ein Protokoll implementiert werden, das die Zuverlässigkeit gewährleistet. Da der UDP-Client im Kontext der Anwendung nur für den Kommunikationsaufbau verwendet werden soll, wird das Send-and-Wait Protokoll implementiert. Dieses Protokoll zeichnet sich durch seine Simplizität in der Implementierung aus. \cite[195]{ethz-vernetzte-systeme}

Für die Implementierung wird ein seperater Thread erstellt, der alle ankommenden Pakete bestätigt und für den Anwender über einen Kanal zur Verfügung stellt. Ebenfalls ist ein Mechanismus implemntiert, der die Verbindung nach einer bestimmten Zeit abbricht, falls keine Antwort vom der Gegenseite empfangen wird.

Da die Kommunikation ebenfalls über das Internet läuft muss sichergestellt werden, dass die Firewalls durch welche die Pakete geroutet werden diese nicht verwerfen. Dazu werden Keep-Alive Pakete versendet, die die Löcher in den Firewalls geöffnet halten. \cite{enginner-man-udp}

Die Verschlüsselung der Kommunikation wird mithilfe der Bibliothek dryoc umgesetzt. Grund dafür ist, dass eine Implementierung einer eigenen Verschlüsselungsmethode ein besonders hohes Risiko bietet. Die Bibliothek dryoc bietet dabei eine simple Schnittstelle, die die Verschlüsselung und Entschlüsselung von Daten ermöglicht.

Es wurde sich expliziert gegen die Verwendung einer zertifizierten Verschlüsselung entscheiden, da die Validität der Zertifikate nicht überprüft werden kann. Stattdessen kann die Kommunikation auch über einen entsprechenden Hash der Schlüssel validiert werden, der über einen seperaten Kanal ausgetauscht werden muss.

Eine Implementierung dieser Funktionalität ist bisher nicht implementiert, wäre aber eine sinnvolle Erweiterung, um Man-in-the-Middle Angriffe zu verhindern.

Die für die Kommunikation verwendete Verschlüsselungsmethode basiert auf ChaCha20-Poly1305. Diese Methode ist in der Lage, die Daten zu verschlüsseln und zu authentifizieren. \cite{google-2015}

Das Protokoll für den Verbindungsaufbau wird mithilfe eines Type-State-Patterns umgesetzt. Das Type-State-Pattern bietet den Vorteil, dass in jedem Zustand nur die erlaubten Aktionen ausgeführt werden können \cite{Apodaca-2023}.

Der erste Status des Protokolls ist eine wartendes UDP-Sockets, dessen Port angefragt werden kann. Sobald beide Parteien ihre Ports ausgetauscht haben, kann ein Verbindungsaufbau versuch gestartet werden. Dafür sendet der Client in einem gegebenen Intervall Nachrichten an die andere Partei. Sobald eine Antwort empfangen wird ist die Verbindung erfolgreich. Andernfalls kann der Verbindungsversuch nach einer gegebenen Zeit automatisch abgebrochen werden.

Im nächsten Schritt werden die Rollen der Parteien festgelegt, da diese für die Verschlüsselung der Kommunikation benötigt werden. Dazu generiert jeder Client eine zufällige Zahl und tauscht diese mit dem Gegenüber aus. Sind beide Zahlen gleich, wird der Prozess wiederholt.

Anschließend werden die öffentlichen Schlüssel ausgetauscht und die darauf basierenden Streams initialisiert.

Falls gewünscht kann nun versucht werden die Kommunikation auf eine TCP-Verbindung umzutellen. Dafür müssen beide Parteien zeitlich synchronisiert werden, da beide TCP-Pakete gleichzeitig abgesendet werden müssen. Das implementierte Protokoll verwendet dazu zwei Ansätze.

Zum einen werden Zeitstempel zwischen den Partein versendet und mit dem Rountrip-Delay verrechnet. Dies wird über mehrere Iterationen wiederholt, da aufgrund von asymmetrischem Netzwerkjitter niemals eine perfekte Synchronisation erreicht werden kann. Schlussendlich wird der Median der gesammelten Werte verwendet. Es gilt festzuhalten, dass falls der Hinweg dauerhaft länger als der Rückweg ist, keine Synchronisation möglich ist.

Alternativ besteht die Möglichkeit über einen externen NTP-Server die Zeit zu synchronisieren.

Sind beide Uhren vermeintlich synchronisiert stimmen beide Clients einen Zeitpunkt ab an dem das TCP-Paket abgesendet wird. Ist der Verbindungsaufbau erfolgreich, wird die UDP-Verbindung geschlossen. Andernfalls kann die UDP-Verbindung weiterhin verwendet werden.

Da das ursprüngliche Send-And-Wait Protokoll für das versenden von Dateien deutlich zulangsam ist, wurde zusätzlich ein weiteres UDP basiertes Protokoll implementiert. Dieses Protokoll verwendet einen Sliding-Window Mechanismus, um so die Anzahl gleichzeitig übertragener Pakete zu erhöhen. Die Anzahl ist dabei variabel und kann durch die Puffergröße bestimmt werden.

Während der Übertragung werden so lange Pakete versendet bis der Sendepuffer voll ist. Der Empfänger bestätigt jeweils das letzte Paket, welches in korrekter Reihenfolge empfangen wurde. Somit werden alle Pakete die kleiner als das zuletzt bestätigte Paket aus dem Sendepuffer enternt. Alle Pakete die nach einer gewissen Zeit nicht bestätigt werden, werden erneut versendet.

\subsection{Datei-Ein-/Ausgabe}
Im Folgenden wird die Funktionsweise des I/O-Moduls beschrieben.

Im Modul Error wurde ein eigener Fehler-Typ implementiert, um die verschiedenen Fehlertypen in den Funktionen zusammenzuführen.

Im Hash-Modul werden Funktionen implementiert, um einen Hash von einer Datei oder einem “u8“-Vektor zu erstellen. Dafür werden die Rust-Bibliotheken “sha2“ und “md-5“ eingebunden. 
Wichtig dabei ist, dass die Puffergröße, die die Datei liest, durch eine Konstante gesteuert wird, sodass zum Hashen nicht der Gesamte Dateiinhalt in den Arbeitsspeicher geladen werden muss.
Außerdem kann durch die Implementierung der Puffergröße als Parameter dieser in Zukunft beispielsweise vom Anwender angepasst werden. Ein großer Puffer bringt Geschwindigkeitsvorteile, da seltener auf den Festplattenspeicher zugegriffen werden muss. Allerdings wird auch deutlich mehr RAM benötigt. Dieser Ansatz wird auch in den anderen Modulen beibehalten, die auf Dateien zugreifen. Ein zu großer Puffer kann auch bei kleinen Dateien zu unnötigen Lese- und Schreibvorgängen führen, was vermieden werden sollte.

Die beiden Module Offer und Order enthalten die Logik, damit die Teilnehmer sich über die Dateien einigen können, die sie austauschen wollen.
Die Nachrichten werden als “u8“-Vektoren kodiert und dem Verbindungsprotokoll zur Verfügung gestellt. Dabei beginnen Offer-Nachrichten mit der Binärsequenz 0000 0001, während Order-Nachrichten mit 0000 0010 beginnen.
Dadurch kann der Client die Nachrichten klassifizieren und den entsprechenden Vorgang starten.

Das Modul File beinhaltet Funktionen, um eine Datei in mehrere Pakete zu zerlegen. Es ist auch möglich, nur einen bestimmten Teil aus einer Datei zu lesen.
Zudem können Datenpakete auch zusammengeführt werden. Dabei wird jedes Paket an die entsprechende Stelle in der neuen Datei geschrieben. Es ist also auch möglich, Dateien in beliebiger Reihenfolge zu schreiben. Falls versucht wird, ein Paket an eine Position in einer Datei zu schreiben, die noch nicht existiert, wird die Datei mit Nullen aufgefüllt, bis die entsprechende Stelle erreicht ist.
Dies würde es zukünftig auch ermöglichen, von mehreren Sendern Pakete zu empfangen und in eine Datei zu schreiben.

Jedes Datenpaket erhält einen Header, der eine variable Länge von bis zu 151 Bytes haben kann. Dort werden alle Informationen übertragen, die zur Validierung des Pakets und zur Zuordnung zu einer Datei benötigt werden.
Im Gegensatz zu Offer und Order beginnt der Header beim Versand von Paketen mit 0.

Die Header-Informationen werden, sobald das Paket in eine Datei geschrieben wurde, in ein Logfile im selben Verzeichnis geschrieben. Dadurch kann die Datei im Nachhinein validiert werden, und es kann überprüft werden, ob ein Paket fehlt. Dafür wird das Logfile ausgelesen und anschließend nach fehlenden Teilen gefiltert.
Die vollständige Logik, um die fehlenden Pakete direkt beim Sender durch eine neue Order zu bestellen, muss noch in die Applikation integriert werden, ist aber bereits im Modul vorhanden.

Als letzten Nachrichtentyp wurde ein Stopsignal implementiert, das den offenen Kanal zwischen Sender und Receiver unterbricht. Dies beginnt mit 0000 0011.

Zum Lesen der Nachrichten, die zwischen Sender und Receiver ausgetauscht werden, sowie dem Auslesen der Logfiles, wird jeweils ein REGEX (Regular Expression) verwendet. Dafür wird die Rust-Bibliothek “regex“ verwendet.
Dadurch können die Nachrichten ohne großen Aufwand ausgelesen und die einzelnen Werte zugeordnet werden.
Da Verschlüsselung bereits durch die Transportprotokolle implementiert wurde, wurde hier darauf verzichtet.

Im letzten Modul General werden alle restlichen Funktionen gelagert, zum Beispiel all jene, die den File-Header für Datenpakete zusammenbauen.
